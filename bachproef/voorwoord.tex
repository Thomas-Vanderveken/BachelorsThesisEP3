%%=============================================================================
%% Voorwoord
%%=============================================================================

\chapter*{\IfLanguageName{dutch}{Woord vooraf}{Preface}}%
\label{ch:voorwoord}
Allereerst wil ik mijn oprechte dank uitspreken aan dhr. Smits voor het aanbieden van dit uitdagende en fascinerende onderwerp voor mijn bachelorproef. Het thema van natuurlijke taalverwerking (NLP) en grote taalmodellen (LLM's) sprak mij bijzonder aan, vooral omdat het in de lessen slechts kort aan bod kwam. De complexiteit van het onderwerp motiveerde mij om dieper in de materie te duiken en mijn kennis uit te breiden, voortbouwend op wat ik tijdens mijn stage heb geleerd.

Daarnaast wil ik mijn ouders hartelijk bedanken voor hun voortdurende steun. Van het transporteren van mij en mijn spullen naar en van mijn kot tot het verzorgen van verse kleren en maaltijden, hun hulp was cruciaal. Hun onvoorwaardelijke steun heeft niet alleen mijn bachelorproef, maar mijn gehele academische traject mogelijk gemaakt. Voor alles wat ze hebben gedaan, ben ik hen enorm dankbaar.

Ik wil ook mijn vrienden bedanken voor hun begrip en geduld tijdens mijn afwezigheid. Jullie hebben me de ruimte gegeven die ik nodig had om me volledig op mijn proefschrift te concentreren, en jullie vriendschap en steun waren een grote troost in drukke tijden.

Mijn mede-kotgenoten verdienen ook een speciale vermelding. Jullie constante aanmoediging en de motiverende gesprekken hielpen me om door te zetten, zelfs op de momenten dat ik het moeilijk vond. Jullie enthousiasme en steun hebben mijn werkproces aanzienlijk verbeterd.

Ten slotte wil ik mijn medestudenten bedanken voor hun waardevolle hulp en advies. Jullie bereidheid om te helpen, zelfs als ik op bepaalde gebieden tekortschiet, en jullie raad hebben bijgedragen aan de verbetering van mijn werk. 

Aan iedereen die op welke manier dan ook heeft bijgedragen aan dit project en mijn academische reis: jullie steun is onmisbaar geweest en ik waardeer dit enorm.
%% Het voorwoord is het enige deel van de bachelorproef waar je vanuit je
%% eigen standpunt (``ik-vorm'') mag schrijven. Je kan hier bv. motiveren
%% waarom jij het onderwerp wil bespreken.
%% Vergeet ook niet te bedanken wie je geholpen/gesteund/... heeft

