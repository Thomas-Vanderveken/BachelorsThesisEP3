%%=============================================================================
%% Conclusie
%%=============================================================================

\chapter{Conclusie}%
\label{ch:conclusie}

% TODO: Trek een duidelijke conclusie, in de vorm van een antwoord op de
% onderzoeksvra(a)g(en). Wat was jouw bijdrage aan het onderzoeksdomein en
% hoe biedt dit meerwaarde aan het vakgebied/doelgroep? 
% Reflecteer kritisch over het resultaat. In Engelse teksten wordt deze sectie
% ``Discussion'' genoemd. Had je deze uitkomst verwacht? Zijn er zaken die nog
% niet duidelijk zijn?
% Heeft het onderzoek geleid tot nieuwe vragen die uitnodigen tot verder 
%onderzoek?





\section{Conclusie}
Deze bachelorproef onderzocht hoe Tekstmining, AI en NLP kunnen worden toegepast om data uit 13F-meldingen van de SEC te extraheren, standaardiseren en integreren in een gestructureerde databank. Het onderzoek begon met het formuleren van een hoofdvraag en enkele deelvragen, die de basis vormden voor de verzameling van de benodigde kennis en de uitvoering van de proof of concept.

De centrale onderzoeksvraag richtte zich op de effectiviteit van AI-technologieën, zoals NLP en Machine Learning, bij het standaardiseren en integreren van vaak inconsistente en ongestructureerde 13F-meldingen van voor 2013. De proof of concept heeft aangetoond dat deze technologieën inderdaad succesvol kunnen worden ingezet, hoewel er enkele beperkingen naar voren kwamen.

Tijdens het onderzoek werden verschillende technieken geëvalueerd voor de gegevensverzameling en extractie. Zo werd regex gebruikt om gestructureerde gegevens uit de headers van de rapporten te halen, terwijl een getraind LLaMA 3.18B-model werd ingezet voor het extraheren en standaardiseren van de tabulaire gegevens. Deze methodologie leidde tot succesvolle extractie en integratie van de gegevens in een relationele databank. De deelvragen hielpen bij het identificeren van de specifieke uitdagingen van de 13F-meldingen voor 2013, de inzet van NLP-technologieën voor het extraheren en standaardiseren van tekstuele gegevens, en de praktische voordelen van het standaardiseren en integreren van deze meldingen met behulp van AI.

De proof of concept toonde aan dat, hoewel de aanpak veelbelovend is, er ook aanzienlijke beperkingen zijn, zoals de beperkte beschikbaarheid van trainingsdata en rekenkracht. Dit leidde tot suboptimale prestaties bij het standaardiseren van gevallen die radicaal afweken van de getrainde subset. Desondanks bieden de behaalde resultaten een solide basis voor verdere ontwikkeling en uitbreiding van deze aanpak. Mogelijke toekomstige uitbreidingen omvatten het gebruik van krachtigere, closed-source modellen zoals GPT, of het trainen van modellen om andere bestandstypen te standaardiseren. Dit zou bijdragen aan een meer uitgebreide transformatie van de EDGAR-databank naar een relationele databank, wat de uitvoering van historische analyses en de ontwikkeling van voorspellende modellen aanzienlijk zou verbeteren.

De meerwaarde van dit onderzoek ligt in het aantonen van de haalbaarheid van het extraheren, standaardiseren en integreren van 13F-meldingen in een databank. Ondanks de beperkingen laat deze studie zien dat met voldoende tijd en middelen, AI-technologieën een cruciale rol kunnen spelen in het efficiënt beheren en analyseren van historische financiële gegevens.










