%%=============================================================================
%% Samenvatting
%%=============================================================================

% TODO: De "abstract" of samenvatting is een kernachtige (~ 1 blz. voor een
% thesis) synthese van het document.
%
% Een goede abstract biedt een kernachtig antwoord op volgende vragen:
%
% 1. Waarover gaat de bachelorproef?
% 2. Waarom heb je er over geschreven?
% 3. Hoe heb je het onderzoek uitgevoerd?
% 4. Wat waren de resultaten? Wat blijkt uit je onderzoek?
% 5. Wat betekenen je resultaten? Wat is de relevantie voor het werkveld?
%
% Daarom bestaat een abstract uit volgende componenten:
%
% - inleiding + kaderen thema
% - probleemstelling
% - (centrale) onderzoeksvraag
% - onderzoeksdoelstelling
% - methodologie
% - resultaten (beperk tot de belangrijkste, relevant voor de onderzoeksvraag)
% - conclusies, aanbevelingen, beperkingen
%
% LET OP! Een samenvatting is GEEN voorwoord!

%%---------- Nederlandse samenvatting -----------------------------------------
%
% TODO: Als je je bachelorproef in het Engels schrijft, moet je eerst een
% Nederlandse samenvatting invoegen. Haal daarvoor onderstaande code uit
% commentaar.
% Wie zijn bachelorproef in het Nederlands schrijft, kan dit negeren, de inhoud
% wordt niet in het document ingevoegd.

\IfLanguageName{english}{%
\selectlanguage{dutch}
\chapter*{Samenvatting}

\selectlanguage{english}
}{}

%%---------- Samenvatting -----------------------------------------------------
% De samenvatting in de hoofdtaal van het document

\chapter*{\IfLanguageName{dutch}{Samenvatting}{Abstract}}


Deze bachelorproef onderzoekt de toepassing van Tekstmining, AI en NLP- technologieën om gegevens uit pre-2013 SEC 13F-meldingen te extraheren, te standaardiseren en te integreren in een gestructureerde databank. Het onderzoek werd gestructureerd in verschillende fasen, te beginnen met het formuleren van een centrale onderzoeksvraag, gevolgd door een uitgebreide literatuurstudie om de uitdagingen van 13F-meldingen te begrijpen en de mogelijkheden van AI in deze context te verkennen. Vervolgens werd een methodologie ontwikkeld, waarbij duidelijke eisen werden gesteld aan de proof of concept (POC).

In de POC werd regex gebruikt om gestructureerde data uit de headers van de 13F-rapporten efficiënt te extraheren, terwijl het LLaMA 3.18B-model werd getraind en ingezet om tabelgegevens te extraheren en te standaardiseren. De implementatie toonde aan dat het mogelijk is om AI voor deze taken in te zetten, maar onthulde ook beperkingen als gevolg van beperkte trainingsdata en rekenkracht, wat de prestaties van het model op niet-standaard gevallen beïnvloedde. De bevindingen suggereren dat, hoewel AI-technieken veelbelovend zijn, verdere verfijning en opschaling nodig zijn voor bredere toepassing.

Uiteindelijk vormt dit onderzoek een sterke basis voor toekomstig werk, waaronder het mogelijke gebruik van geavanceerdere modellen zoals GPT en het verkennen van andere bestandstypen. De studie benadrukt het significante potentieel van AI om de verwerking en analyse van historische financiële gegevens te verbeteren, wat de weg vrijmaakt voor efficiënter gegevensbeheer en verbeterde voorspellende modellen.

De meerwaarde van dit onderzoek is vooral relevant voor financiële instellingen, onderzoekers en data-analisten die werken met historische financiële gegevens, met name die van de SEC 13F-meldingen. Door de ontwikkeling van een proof of concept die aantoont dat AI-technologieën zoals NLP en machine learning effectief kunnen worden ingezet voor het standaardiseren en integreren van dergelijke gegevens, biedt dit onderzoek inzichten en tools die de efficiëntie en nauwkeurigheid van gegevensverwerking aanzienlijk kunnen verbeteren.