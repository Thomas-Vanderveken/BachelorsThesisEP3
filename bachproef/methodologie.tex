%%=============================================================================
%% Methodologie
%%=============================================================================

\chapter{\IfLanguageName{dutch}{Methodologie}{Methodology}}%
\label{ch:methodologie}

%% : In dit hoofstuk geef je een korte toelichting over hoe je te werk bent
%% gegaan. Verdeel je onderzoek in grote fasen, en licht in elke fase toe wat
%% de doelstelling was, welke deliverables daar uit gekomen zijn, en welke
%% onderzoeksmethoden je daarbij toegepast hebt. Verantwoord waarom je
%% op deze manier te werk gegaan bent.
%% 
%% Voorbeelden van zulke fasen zijn: literatuurstudie, opstellen van een
%% requirements-analyse, opstellen long-list (bij vergelijkende studie),
%% selectie van geschikte tools (bij vergelijkende studie, "short-list"),
%% opzetten testopstelling/PoC, uitvoeren testen en verzamelen
%% van resultaten, analyse van resultaten, ...
%%
%% !!!!! LET OP !!!!!
%%
%% Het is uitdrukkelijk NIET de bedoeling dat je het grootste deel van de corpus
%% van je bachelorproef in dit hoofstuk verwerkt! Dit hoofdstuk is eerder een
%% kort overzicht van je plan van aanpak.
%%
%% Maak voor elke fase (behalve het literatuuronderzoek) een NIEUW HOOFDSTUK aan
%% en geef het een gepaste titel.

Dit hoofdstuk geeft een overzicht van de methodologie die is gebruikt om dit onderzoek uit te voeren en de Proof of Concept (POC) te creëren. De tekst biedt een uitgebreide analyse van het belang van elke fase van het onderzoek en licht de redenering achter de gekozen methodologieën en benaderingen toe. Dit hoofdstuk maakt duidelijk hoe de gekozen benaderingen helpen om de onderzoeksdoelen te bereiken door een goed georganiseerd overzicht te bieden. Het belang van elke fase wordt benadrukt, waardoor inzicht wordt verkregen in de achterliggende gedachte van de beslissingen die tijdens het onderzoeksproces zijn genomen.

\section{Fase 1 - Literatuur studie}
De eerste fase van dit onderzoek bestond uit een uitgebreid onderzoek van bestaande literatuur. Het doel van deze fase was om een grondig begrip te krijgen van de concepten en technologieën die gebruikt zouden worden bij de implementatie van de Proof of Concept (POC). De bovengenoemde stap omvatte een uitgebreide analyse van verschillende publicaties, papers, blogs en handleidingen om relevante toepassingen en benaderingen te ontdekken. De belangrijkste onderwerpen die in deze fase werden onderzocht waren Text Mining, Natural Language Processing (NLP) en Database Management Systemen (DBMS). Deze fase zal
6 weken in beslag nemen en zal als resultaat een literatuurstudie zijn die van groot belang is om het verdere verloop van het onderzoek te begrijpen.
\section{Fase 2 - Requirements analyse}
Het doel van deze fase is het opstellen van de criteria waaraan de POC moet voldoen en het identificeren van de specifieke onderdelen die erin moeten zitten om als succesvol te worden beschouwd. Deze fase zal een week duren om succesvol te voltooien. Volgens \autocite{Achimigu2014} is de MoSCoW-techniek een methode die kan worden gebruikt om prioriteit toe te kennen aan leveringen. Deze operatie wordt uitgevoerd met behulp van de MoSCoW-methode. D.w.z. er kan ook een prioriteit worden toegekend aan de verschillende benodigdheden als ze op deze manier worden georganiseerd. Het eindproduct van dit deel van het onderzoek was een geprioriteerde lijst van wat wel en niet nodig was om een succesvol proof of concept te genereren. 

\section{Fase 3 - POC}  
In deze fase wordt het Proof of Concept (PoC) uitgevoerd, waarbij er in verschillende stappen de haalbaarheid en effectiviteit van de voorgestelde oplossingen zullen testen. De belangrijkste stappen omvatten de implementatie van de geselecteerde technieken, het verzamelen en voorbereiden van de benodigde data, en het configureren van de database. Vervolgens wordt de PoC geëvalueerd op basis van de vastgestelde criteria om de resultaten te analyseren en verdere verfijningen aan te brengen. Deze fase zal naar verwachting 5 weken in beslag nemen.
\subsection{Dataset creatie}
Tijdens deze fase hebben wordt er een dataset gecreëerd met de 13F-dossiers als basis, die de basis vormde voor de constructie van het Proof of Concept (POC). De informatie werd zorgvuldig samengesteld door pertinente financiële gegevens uit de 13F-papieren te halen, waarbij gegarandeerd werd dat de informatie geordend en geformatteerd werd op een manier die geschikt is voor latere analyse en verwerking binnen het POC-kader.

\subsection{Vergelijking technieken}
In deze sectie wordt een gedetailleerde vergelijking gepresenteerd van technieken binnen de domeinen van Natuurlijke Taalverwerking (NLP), Machine Learning (ML), en Data Mining (DM). Elke techniek heeft haar eigen unieke voor- en nadelen, afhankelijk van het toepassingsgebied en de specifieke doelen die nagestreefd worden.

\subsection{Databank}
Voor dit project wordt PostgreSQL ingezet als database-oplossing om te voldoen aan de eisen voor gegevensbeheer. PostgreSQL biedt robuuste mogelijkheden voor het opslaan, verwerken en opvragen van grote hoeveelheden financiële gegevens. De database wordt geoptimaliseerd voor het effectief beheren van semi-gestructureerde gegevens die voortkomen uit de NLP-taken.

\subsection{Implementatie}
In dit gedeelte wordt het Proof of Concept (POC) uitgevoerd om de effectiviteit van verschillende benaderingen te evalueren voor het verwerken en analyseren van de financiële gegevens in de 13F-dossiers. Hierbij zullen er diverse tools en frameworks onderzoeken en testen om essentiële NLP-taken uit te voeren, zoals tekstextractie, Named Entity Recognition (NER), en informatie herwinning. Het POC richt zich op het demonstreren van de haalbaarheid en effectiviteit van de gekozen methoden in het project.

\subsection{Analyse van de resultaten}
In deze fase zullen de resultaten van het POC analyseren, waarbij de prestaties van de vergeleken technieken en de efficiëntie van de database-oplossing worden geëvalueerd. Deze analyse biedt inzicht in de verdere toepassing van de resultaten en mogelijke vervolgstappen binnen het project. %TODO

\section{Fase - 4}
In de loop van deze laatste fase zal er een extra week worden gereserveerd voor het opstellen van een conclusie en abstract over het verloop van het onderzoek. Daarnaast is er in de laatste twee weken van het onderzoek tijd voorzien om de scriptie voor deze bachelorproef af te ronden.